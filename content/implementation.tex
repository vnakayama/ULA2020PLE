\section*{2. Descriçāo da Implementaçāo}

\subsection*{2.1 Funções Implementadas}

\\\subsubsection*{2.1.1. Parte Lógica}
\newline
Foram implementadas quatro funções lógicas : \textit{AND}, \textit{OR}, \textit{NOT} e \textit{XOR}, aplicadas bit-a-bit no vetor de entrada. Essas operações são identificadas através de um vetor de seleção de três bits, com os códigos abaixo.

\begin{table}[h]
    \centering
    \begin{tabular}{|c|c|}
     \hline \textbf{Operaçāo} & \textbf{Vetor} \\
     \hline AND & 000\\
     \hline OR & 001\\
     \hline NOT & 010\\
     \hline XOR & 011\\
    \hline

\end{tabular}
    \caption{Funções lógicas e correspondente no vetor de seleção \texttt{op}.}
    \label{tab:t1}
\end{table}

As operações foram efetivadas da seguinte maneira: criou-se uma entidade \texttt{ULA}\textbf{***APENDICE***} e um vetor de 3 bits, chamado de \texttt{op},  responsável pela seleçāo da operaçāo a ser realizada. Desta maneira, a partir da inserçāo de dois vetor de entrada de 4 bits e da operaçāo selecionada pelo vetor \texttt{op}, a \texttt{ULA} realiza as funções lógicas citadas acima. 


\\\subsubsection*{2.1.2. Parte Aritmética}

As Funções Aritméticas implementadas foram: \textit{Soma}, \textit{Subtraçāo}, \textit{Multiplicaçāo} e \textit{Complemento de 2}. Também podemos ver na tabela abaixo a maneira que relacionamos cada uma dessas operações a um número no vetor \texttt{op} de 3 bits.

\begin{table}[h]
    \centering
    \begin{tabular}{|c|c|}
     \hline \textbf{Operaçāo} & \textbf{Vetor} \\
     \hline Soma & 100\\
     \hline Subtração & 101\\
     \hline Multiplicação & 110\\
     \hline Complemento de 2 & 111\\
    \hline

\end{tabular}
    \caption{Funções aritméticas e correspondente no vetor de seleção \texttt{op}.}
    \label{tab:t2}
\end{table}

\subsection*{2.2. Módulos Básicos}

\subsubsection*{2.2.1. Somador de 1 bit}

\subsubsection*{2.2.3. \textit{Full-adder}}

\subsubsection*{2.2.3. Contador}

\subsection*{2.3. Funções Lógicas}
\subsubsection*{2.3.1. \textit{AND}}
\subsubsection*{2.3.2. \textit{OR}}
\subsubsection*{2.3.3. \textit{NOT}}
\subsubsection*{2.3.4. \textit{XOR}}

\subsection*{2.4. Funções Aritméticas}

\subsubsection*{2.4.1. Funçāo Soma}

A função soma foi a primeira Função Aritmética a ser implementada na \textbf{ULA}. Essa escolha se deu por conta das Fun


\subsubsection*{2.4.2. Funçāo Complemento de 2}

\subsubsection*{2.4.3. Funçāo Subtraçāo}

Como a \textit{Funçāo Soma} foi a primeira ser implementada, implementamos \textit{Funçāo Complemento de 2} logo em seguida justamente para tornarmos mais fácil a implementaçāo da \textit{Funçāo Subtraçāo}. Pois dessa maneira, podemos simplesmente utilizar esses dois módulos para implementar a subtraçāo, visto que esta é a soma de um número com o Complemento de 2 do outro, sendo este último encontrado com módulo negativo. Da mesma maneira que fizemos na implementaçāo da \textit{Função Soma}, aqui fazemos um Port Map com entradas para os dois números a serem subtraídos, bem como uma saída para o resultado. \textbf{***APENDICE***} Por fim, foi considerado nesse módulo tanto o carry out quanto o overflow.

O funcionamento da \textbf{Função Subtração} se dá da seguinte maneira: primeiramente a entrada \textbf{b} do subtrator é convertida para complemento de dois e renomeada como \textbf{c2b}. Por fim, o valor de c2b é somado ao vetor da entrada \textbf{a}, tendo assim o resultado retornado pela função.


\subsubsection*{2.4.4. Funçāo Multiplicaçāo}

% \inputminted{vhdl}{code/example.vhd}

% Incluir prints das funções (talvez nem seja necessário, acho que só o código já seria legal)
%
% \begin{figure}[H]
%     \centering
%     \includegraphics{}
%     \caption{Caption}
%     \label{fig:my_label}
% \end{figure}